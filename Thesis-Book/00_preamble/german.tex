% =============================================================================
% DEUTSCHE SPRACHEINSTELLUNGEN FÜR THESIS BOOK
% =============================================================================
% Diese Datei enthält alle deutschen Sprachbefehle und Labels
% =============================================================================

% =============================================================================
% SPRACHBEFEHLE
% =============================================================================

% Grundlegende Labels
\newcommand{\toObtainLabel}{zur Erlangung des akademischen Grades}
\newcommand{\submittedByLabel}{Eingereicht von:}
\newcommand{\matNumberLabel}{Personenkennzeichen:}
\newcommand{\dateLabel}{Datum:}
\newcommand{\advisorLabel}{Betreut von:}
\newcommand{\literatureLabel}{Literaturverzeichnis}
\newcommand{\abstractLabel}{Kurzfassung}
\newcommand{\contentsLabel}{Inhaltsverzeichnis}
\newcommand{\listoffiguresLabel}{Abbildungsverzeichnis}
\newcommand{\listoftablesLabel}{Tabellenverzeichnis}
\newcommand{\acronymLabel}{Akronyme}

% Kapitel-Labels
\newcommand{\introductionLabel}{Einleitung und Problemhintergrund}
\newcommand{\relatedWorkLabel}{Stand des Wissens / Stand der Technik}
\newcommand{\researchQuestionLabel}{Wissenschaftliche Fragestellung}
\newcommand{\methodologyLabel}{Forschungsmethodik}
\newcommand{\resultsLabel}{Ergebnisse}
\newcommand{\conclusionLabel}{Schlussfolgerungen und Ausblick}

% Formatierungs-Labels
\newcommand{\figureLabel}{Abb.}
\newcommand{\tableLabel}{Tab.}
\newcommand{\equationLabel}{Gl.}
\newcommand{\pageLabel}{Seite}
\newcommand{\chapterLabel}{Kapitel}
\newcommand{\sectionLabel}{Abschnitt}

% LaTeX-Standard-Befehle überschreiben
\renewcommand{\tablename}{\tableLabel}
\renewcommand{\figurename}{\figureLabel}
\renewcommand{\thetable}{\arabic{table}}
\renewcommand{\thefigure}{\arabic{figure}}

% =============================================================================
% DEUTSCHE ZITIERSTILE
% =============================================================================

% Deutsche Anführungszeichen
\newcommand{\germanquote}[1]{„#1"}

% Deutsche Datumsformatierung
\newcommand{\germandate}[3]{#1. #2 #3}

% Deutsche Zahlenformatierung
\newcommand{\germannumber}[1]{\numprint{#1}}

% Deutsche Literaturverweise
\newcommand{\germanref}[1]{(\citeauthor{#1} \citeyear{#1})}
\newcommand{\germanrefs}[2]{(\citeauthor{#1} \citeyear{#1}, \citeauthor{#2} \citeyear{#2})}

% =============================================================================
% DEUTSCHE FORMATIERUNGEN
% =============================================================================

% Deutsche Absatzformatierung
\newcommand{\germanparagraph}{%
    \setlength{\parindent}{0.4cm}%
    \setlength{\parskip}{0pt}%
}

% Deutsche Aufzählungsformatierung
\newcommand{\germanitemize}{%
    \begin{itemize}[leftmargin=0.63cm, label=\textbullet]%
}

% Deutsche Tabellenformatierung
\newcommand{\germantable}[3]{%
    \begin{table}[htbp]%
    \centering%
    \caption{#1}%
    \label{#2}%
    \begin{tabular}{#3}%
    \toprule%
}

% Deutsche Abbildungsformatierung
\newcommand{\germanfigure}[2]{%
    \begin{figure}[htbp]%
    \centering%
    \includegraphics[width=\textwidth]{#1}%
    \caption{#2}%
    \label{fig:#1}%
}

% =============================================================================
% DEUTSCHE VALIDIERUNG
% =============================================================================

% Überprüfung der deutschen Spracheinstellungen
\makeatletter
\@ifpackageloaded{babel}{%
    \typeout{Deutsche Spracheinstellungen korrekt geladen}%
}{%
    \typeout{WARNUNG: babel-Package nicht geladen!}%
}
\makeatother

% Überprüfung der deutschen Rechtschreibung
\makeatletter
\@ifpackageloaded{biblatex}{%
    \typeout{Deutsche Literaturverwaltung korrekt geladen}%
}{%
    \typeout{WARNUNG: biblatex-Package nicht geladen!}%
}
\makeatother