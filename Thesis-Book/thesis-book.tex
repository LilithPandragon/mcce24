% =============================================================================
% Masterarbeit LaTeX-Vorlage - Angepasst an Word-Vorlage
% =============================================================================
% Diese Vorlage entspricht exakt den Formatierungsvorgaben der Word-Vorlage
% für Masterarbeiten an der Hochschule Burgenland 2025-10-12
% =============================================================================

\documentclass[11pt,a4paper]{report}

% =============================================================================
% PACKAGES UND GRUNDEINSTELLUNGEN
% =============================================================================

% Zeichenkodierung
\usepackage[utf8]{inputenc}
\usepackage[T1]{fontenc}

% Spracheinstellung
\usepackage[english]{babel}

% Schriftart: Times New Roman
\usepackage{mathptmx}
\usepackage[scaled=0.9]{helvet}
\usepackage{courier}

% Seitenlayout - Exakt nach Word-Vorlage
\usepackage[
    left=3.0cm,
    right=3.0cm,
    top=2.5cm,
    bottom=3.0cm,
    a4paper
]{geometry}

% Zeilenabstand: Genau 12pt (entspricht 1,09 in LaTeX)
\usepackage{setspace}
\setstretch{1.09}

% Textausrichtung: Blocksatz
\usepackage{ragged2e}
\justifying

% Überschriften-Formatierung
\usepackage{titlesec}
\usepackage{sectsty}

% Tabellen und Abbildungen
\usepackage{booktabs}
\usepackage{tabularx}
\usepackage{graphicx}
\usepackage{caption}
\usepackage{subcaption}

% Aufzählungen
\usepackage{enumitem}

% Akronyme
\usepackage[printonlyused, nohyperlinks]{acronym}

% Mathematik
\usepackage{amsmath}
\usepackage{amssymb}

% Literaturverwaltung
\usepackage[
    style=authoryear-icomp,
    sorting=nty,
    backend=biber,
    language=english,
    sortlocale=en-US,
    natbib=true,
    maxbibnames=1000,
]{biblatex}
\addbibresource{references.bib}

% Hyperlinks
\usepackage{hyperref}
\hypersetup{
    colorlinks=true,
    linkcolor=black,
    citecolor=black,
    urlcolor=blue,
    bookmarksnumbered=true,
    pdfstartview={Fit},
    pdfpagemode=UseOutlines,
}

% =============================================================================
% FORMATIERUNGEN NACH WORD-VORLAGE 2025-10-12
% =============================================================================

% Hauptschrift: Times New Roman, 11pt
\renewcommand{\normalsize}{\fontsize{11pt}{12pt}\selectfont}
\renewcommand{\large}{\fontsize{12pt}{14pt}\selectfont}
\renewcommand{\Large}{\fontsize{14pt}{16pt}\selectfont}
\renewcommand{\LARGE}{\fontsize{16pt}{18pt}\selectfont}

% Überschriften Ebene 1 (Hauptüberschriften)
\titleformat{\chapter}
    {\fontsize{11pt}{12pt}\selectfont\bfseries\MakeUppercase}
    {\thechapter}
    {20pt}
    {\fontsize{11pt}{12pt}\selectfont\bfseries\MakeUppercase}

\titlespacing{\chapter}{0pt}{36pt}{6pt}

% Überschriften Ebene 2
\titleformat{\section}
    {\fontsize{11pt}{12pt}\selectfont\bfseries}
    {\thesection}
    {1em}
    {\fontsize{11pt}{12pt}\selectfont\bfseries}

\titlespacing{\section}{0pt}{12pt}{6pt}

% Überschriften Ebene 3
\titleformat{\subsection}
    {\fontsize{11pt}{12pt}\selectfont\bfseries}
    {\thesubsection}
    {1em}
    {\fontsize{11pt}{12pt}\selectfont\bfseries}

\titlespacing{\subsection}{0pt}{12pt}{6pt}

% Absatzformatierung
\setlength{\parindent}{0.4cm}  % Einzug für Folgeabsätze: 0,4 cm
\setlength{\parskip}{0pt}      % Kein Abstand zwischen Absätzen

% Erster Absatz nach Überschrift: Keine Einrückung
\makeatletter
\renewcommand{\@afterheading}{%
    \@nobreaktrue
    \everypar{%
        \if@nobreak
            \@nobreakfalse
            \clubpenalty \@M
            \setlength{\parindent}{0pt}%
        \else
            \setlength{\parindent}{0.4cm}%
        \fi
    }%
}
\makeatother

% Aufzählungen
\setlist[itemize]{
    leftmargin=0.63cm,
    label=\textbullet,
    itemsep=0pt,
    parsep=0pt,
    topsep=0pt
}

% Tabellenformatierung
\captionsetup[table]{
    font=small,
    labelfont=bf,
    textfont=normalfont,
    justification=justified,
    singlelinecheck=false,
    position=above,
    skip=6pt
}

% Tabellenformatierung wird in Preamble-Datei definiert

% Abbildungsformatierung
\captionsetup[figure]{
    font=small,
    labelfont=bf,
    textfont=normalfont,
    justification=justified,
    singlelinecheck=false,
    position=below,
    skip=6pt
}

% Abbildungsformatierung wird in Preamble-Datei definiert

% Gleichungen
\makeatletter
\renewcommand{\@eqnnum}{%
    \ifx\@empty\@eqnnum@prefix\else\@eqnnum@prefix\fi
    \ifx\@empty\@eqnnum@suffix\else\@eqnnum@suffix\fi
    \hfill\@eqnnum@number
}
\makeatother

% Literaturverzeichnis
\renewcommand{\bibfont}{\fontsize{10pt}{11pt}\selectfont}
\setlength{\bibhang}{0.4cm}  % Hängender Einzug: 0,4 cm ab zweiter Zeile

% =============================================================================
% DOKUMENTDATEN
% =============================================================================

% Diese Werte müssen in yourData.tex angepasst werden
% Alle Daten sind jetzt in den Sprachdateien (german.tex / english.tex)
% Diese Datei ist leer, da alle \newcommand-Definitionen in der Preamble stehen

% Spracheinstellungen laden
% =============================================================================
% DEUTSCHE SPRACHEINSTELLUNGEN FÜR THESIS BOOK
% =============================================================================
% Diese Datei enthält alle deutschen Sprachbefehle und Labels
% =============================================================================

% =============================================================================
% SPRACHBEFEHLE
% =============================================================================

% Grundlegende Labels
\newcommand{\toObtainLabel}{zur Erlangung des akademischen Grades}
\newcommand{\submittedByLabel}{Eingereicht von:}
\newcommand{\matNumberLabel}{Personenkennzeichen:}
\newcommand{\dateLabel}{Datum:}
\newcommand{\advisorLabel}{Betreut von:}
\newcommand{\literatureLabel}{Literaturverzeichnis}
\newcommand{\abstractLabel}{Kurzfassung}
\newcommand{\contentsLabel}{Inhaltsverzeichnis}
\newcommand{\listoffiguresLabel}{Abbildungsverzeichnis}
\newcommand{\listoftablesLabel}{Tabellenverzeichnis}

% Kapitel-Labels
\newcommand{\introductionLabel}{Einleitung und Problemhintergrund}
\newcommand{\relatedWorkLabel}{Stand des Wissens / Stand der Technik}
\newcommand{\researchQuestionLabel}{Wissenschaftliche Fragestellung}
\newcommand{\methodologyLabel}{Forschungsmethodik}
\newcommand{\resultsLabel}{Ergebnisse}
\newcommand{\conclusionLabel}{Schlussfolgerungen und Ausblick}

% Formatierungs-Labels
\newcommand{\figureLabel}{Abb.}
\newcommand{\tableLabel}{Tab.}
\newcommand{\equationLabel}{Gl.}
\newcommand{\pageLabel}{Seite}
\newcommand{\chapterLabel}{Kapitel}
\newcommand{\sectionLabel}{Abschnitt}

% =============================================================================
% DEUTSCHE ZITIERSTILE
% =============================================================================

% Deutsche Anführungszeichen
\newcommand{\germanquote}[1]{„#1"}

% Deutsche Datumsformatierung
\newcommand{\germandate}[3]{#1. #2 #3}

% Deutsche Zahlenformatierung
\newcommand{\germannumber}[1]{\numprint{#1}}

% Deutsche Literaturverweise
\newcommand{\germanref}[1]{(\citeauthor{#1} \citeyear{#1})}
\newcommand{\germanrefs}[2]{(\citeauthor{#1} \citeyear{#1}, \citeauthor{#2} \citeyear{#2})}

% =============================================================================
% DEUTSCHE FORMATIERUNGEN
% =============================================================================

% Deutsche Absatzformatierung
\newcommand{\germanparagraph}{%
    \setlength{\parindent}{0.4cm}%
    \setlength{\parskip}{0pt}%
}

% Deutsche Aufzählungsformatierung
\newcommand{\germanitemize}{%
    \begin{itemize}[leftmargin=0.63cm, label=\textbullet]%
}

% Deutsche Tabellenformatierung
\newcommand{\germantable}[3]{%
    \begin{table}[htbp]%
    \centering%
    \caption{#1}%
    \label{#2}%
    \begin{tabular}{#3}%
    \toprule%
}

% Deutsche Abbildungsformatierung
\newcommand{\germanfigure}[2]{%
    \begin{figure}[htbp]%
    \centering%
    \includegraphics[width=\textwidth]{#1}%
    \caption{#2}%
    \label{fig:#1}%
}

% =============================================================================
% DEUTSCHE VALIDIERUNG
% =============================================================================

% Überprüfung der deutschen Spracheinstellungen
\makeatletter
\@ifpackageloaded{babel}{%
    \typeout{Deutsche Spracheinstellungen korrekt geladen}%
}{%
    \typeout{WARNUNG: babel-Package nicht geladen!}%
}
\makeatother

% Überprüfung der deutschen Rechtschreibung
\makeatletter
\@ifpackageloaded{biblatex}{%
    \typeout{Deutsche Literaturverwaltung korrekt geladen}%
}{%
    \typeout{WARNUNG: biblatex-Package nicht geladen!}%
}
\makeatother

% =============================================================================
% DOKUMENT BEGINN
% =============================================================================

\begin{document}

% Titelseite
\thispagestyle{empty}
\begin{center}
    \vspace*{2cm}
    
    % Titel: Times New Roman, 16pt, Zeilenabstand 18pt
    {\fontsize{16pt}{18pt}\selectfont\bfseries\yourThesisTitle}
    
    \vspace{3cm}
    
    % Autoren: Times New Roman, 12pt, Zeilenabstand 14pt
    {\fontsize{12pt}{14pt}\selectfont\yourNameInclTitle}
    
    \vspace{1cm}
    
    % Institution: Times New Roman, 10pt, Zeilenabstand 11pt
    {\fontsize{10pt}{11pt}\selectfont\universityCityCountry}
    
    \vspace{2cm}
    
    % Zusätzliche Informationen
    {\fontsize{10pt}{11pt}\selectfont
    \typeOfWork\\
    \studyProgram\\
    \thesisDate
    }
\end{center}

\clearpage

% Kurzfassung
% Kurzfassung Formatierung: Text direkt nach "KURZFASSUNG:" in derselben Zeile
% =============================================================================
% KURZFASSUNG (ABSTRACT) - DEUTSCH
% =============================================================================
% Formatierung:
% - Schriftart: Times New Roman, 11pt
% - Zeilenabstand: Genau 12pt
% - Länge: 12-15 Zeilen
% - Aufbau: "KURZFASSUNG:" Text beginnt in selber Zeile 
% =============================================================================

% Der Text sollte 12-15 Zeilen umfassen

\textbf{KURZFASSUNG:} Dies ist ein Beispieltext für die Kurzfassung. Hier sollte der Inhalt der Masterarbeit in komprimierter
Form dargestellt werden. Die Kurzfassung sollte die wichtigsten Aspekte der Arbeit umfassen:

TODO

\begin{itemize}[leftmargin=0.63cm, label=\textbullet]
    \item Problemstellung und Motivation
    \item Methodischer Ansatz
    \item Hauptergebnisse
    \item Schlussfolgerungen
\end{itemize}

Der Text sollte präzise und verständlich formuliert sein, sodass auch fachfremde Leser die Bedeutung der
Arbeit erfassen können. Die Kurzfassung dient als erster Eindruck der Arbeit und sollte daher sorgfältig verfasst werden.



\clearpage

\clearpage

% Inhaltsverzeichnis
\tableofcontents
\clearpage

% =============================================================================
% HAUPTKAPITEL
% =============================================================================

% Kapitel 1: Einleitung und Problemhintergrund


\chapter{\introductionlabel}\label{cha:introduction}

Dies ist ein Test-Text mit einer Zitation~\cite{mustermann2023}.

TODO



% Kapitel 2: Stand des Wissens / Stand der Technik


\chapter{\relatedWorklabel} \label{cha:relatedWork}

TODO



% Kapitel 3: Wissenschaftliche Fragestellung


\chapter{\researchQuestionlabel}\label{cha:researchQuestion}

TODO



% Kapitel 4: Forschungsmethodik


\chapter{\methodologicalApproachlabel}\label{cha:methodologicalApproach}

TODO



% Kapitel 5: Ergebnisse
% =============================================================================
% KAPITEL 5: ERGEBNISSE
% =============================================================================
% Formatierung:
% - Überschrift: Times New Roman, 11pt, Großbuchstaben
% - Abstand vor: 36pt, Abstand nach: 6pt
% - Erster Absatz: Keine Einrückung, linksbündig
% - Folgeabsätze: 0,4cm Einzug, Blocksatz
% =============================================================================

\chapter{Ergebnisse}\label{cha:results}

TODO

% Erster Absatz nach Überschrift: Keine Einrückung
Dieses Kapitel präsentiert die Ergebnisse der durchgeführten Untersuchung. Hier werden die gewonnenen Erkenntnisse systematisch dargestellt und analysiert.

% Folgeabsätze: 0,4cm Einzug
Die Ergebnisse werden strukturiert präsentiert und umfassen verschiedene Aspekte:

\begin{itemize}[leftmargin=0.63cm, label=\textbullet]
    \item Darstellung der Hauptergebnisse
    \item Analyse der Daten
    \item Interpretation der Befunde
    \item Diskussion der Implikationen
\end{itemize}

Die Präsentation der Ergebnisse erfolgt objektiv und nachvollziehbar. Dabei werden sowohl positive als auch negative Aspekte berücksichtigt.

% Beispiel für Literaturverweis
Die Ergebnisse bestätigen die Annahmen aus~\cite{beispiel2024}.

% Beispiel für Tabelle 
\begin{table}[htbp]
    \centering
    \caption{Zusammenfassung der Ergebnisse}
    \label{tab:ergebnisse}
        \begin{tabular}{lcc}
        \toprule
        Kategorie & Anzahl & Prozent \\
        \midrule
        Positiv & 45 & 75\% \\
        Neutral & 10 & 17\% \\
        Negativ & 5 & 8\% \\
        \bottomrule
        \end{tabular}
\end{table}

Die Tabelle zeigt eine Zusammenfassung der wichtigsten Ergebnisse.

% Beispiel für Abbildung 
\begin{figure}[htbp]
    \centering
    % \includegraphics[width=0.8\textwidth]{ergebnisse-diagramm.png}
    \caption{Visualisierung der Ergebnisse}
    \label{fig:ergebnisse}
\end{figure}

Die Abbildung veranschaulicht die wichtigsten Ergebnisse der Untersuchung.

% Beispiel für Gleichung 
\begin{equation}
    \label{eq:ergebnis}
    R = \frac{\sum_{i=1}^{n} x_i}{n}
\end{equation}

Die Gleichung zeigt die Berechnung des Durchschnittswerts der Ergebnisse.


% Kapitel 6: Schlussfolgerungen und Ausblick
% =============================================================================
% KAPITEL 6: SCHLUSSFOLGERUNGEN UND AUSBLICK
% =============================================================================
% Formatierung:
% - Überschrift: Times New Roman, 11pt, Großbuchstaben
% - Abstand vor: 36pt, Abstand nach: 6pt
% - Erster Absatz: Keine Einrückung, linksbündig
% - Folgeabsätze: 0,4cm Einzug, Blocksatz
% =============================================================================

\chapter{\conclusionLabel}\label{cha:conclusion}

TODO 

% Erster Absatz nach Überschrift: Keine Einrückung
Dieses Kapitel fasst die wichtigsten Erkenntnisse der Masterarbeit zusammen und gibt einen Ausblick auf mögliche zukünftige Entwicklungen. 
Hier werden die Ergebnisse bewertet und in den größeren Kontext eingeordnet.

% Folgeabsätze: 0,4cm Einzug
Die Schlussfolgerungen basieren auf den gewonnenen Erkenntnissen und umfassen verschiedene Aspekte:

\begin{itemize}[leftmargin=0.63cm, label=\textbullet]
    \item Zusammenfassung der Haupterkenntnisse
    \item Bewertung der Ergebnisse
    \item Einordnung in den Forschungskontext
    \item Ausblick auf zukünftige Entwicklungen
\end{itemize}

Die Schlussfolgerungen werden präzise formuliert und durch die Ergebnisse der Untersuchung gestützt. 
Dabei werden auch Limitationen und offene Fragen diskutiert.

% Beispiel für Literaturverweis
Die Ergebnisse bestätigen die Annahmen aus~\cite{mustermann2023} und erweitern den Kenntnisstand in diesem Bereich.

% Beispiel für Tabelle 
\begin{table}[htbp]
    \centering
    \caption{Zusammenfassung der Schlussfolgerungen}
    \label{tab:schlussfolgerungen}
        \begin{tabular}{lcc}
        \toprule
        Aspekt & Bewertung & Ausblick \\
        \midrule
        Methodik & Erfolgreich & Weiterentwicklung \\
        Ergebnisse & Positiv & Anwendung \\
        Limitationen & Identifiziert & Überwindung \\
        \bottomrule
        \end{tabular}
\end{table}

Die Tabelle zeigt eine Zusammenfassung der wichtigsten Schlussfolgerungen.

% Beispiel für Abbildung 
\begin{figure}[htbp]
    \centering
    % \includegraphics[width=0.8\textwidth]{ausblick-diagramm.png}
    \caption{Ausblick auf zukünftige Entwicklungen}
    \label{fig:ausblick}
\end{figure}

Die Abbildung veranschaulicht mögliche zukünftige Entwicklungen in diesem Bereich.




% =============================================================================
% LITERATURVERZEICHNIS und Akronyme
% =============================================================================

\phantomsection
\addcontentsline{toc}{chapter}{\literatureLabel}
\printbibliography[title=\literatureLabel]

% Akronyme
\phantomsection
\addcontentsline{toc}{chapter}{Akronyme}
\chapter*{Akronyme}

\begin{acronym}[MPJPEG]
    \acro{MPJPEG}[MPJPEG]{Multipart JPEG}
    \acro{NAL}[NAL]{Network Abstraction Layer}
    \acroplural{NAL}[NALs]{Network Abstraction Layers}
    \acro{IoT}[IoT]{Internet of Things}
    \acro{AI}[AI]{Artificial Intelligence}
    \acro{ML}[ML]{Machine Learning}
    \acro{DL}[DL]{Deep Learning}
    \acro{NIST}[NIST]{National Institute of Standards and Technology}
    \acro{SaaS}[SaaS]{Software as a Service}
    \acro{PaaS}[PaaS]{Platform as a Service}
    \acro{IaaS}[IaaS]{Infrastructure as a Service}
\end{acronym}



\end{document}